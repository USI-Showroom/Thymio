\documentclass[12pt]{article}
\usepackage[margin=2cm]{geometry} 
\usepackage{titling}
\usepackage{graphicx}
\usepackage{float}
\usepackage[hidelinks]{hyperref}
\usepackage[italian]{babel}

\setlength\parindent{0pt}
\setlength{\parskip}{1em}
\setlength{\droptitle}{-2cm}

\title{Istruzioni d'uso Thymio}
\author{Università della Svizzera Italiana}
\date{Versione \today \ (Thymio Suite 2.0.0)}


\begin{document}
\maketitle
\tableofcontents
\newpage


\section{Installazione}\label{installation}

	\subsection{Windows}
	
		Recarsi all'indirizzo \url{https://www.thymio.org/program-2/} e scaricare il software Thymio Suite cliccando sul bottone \texttt{Download} sotto il logo Windows. Prestare attenzione a scegliere la versione adatta al proprio sistema (32/64 bit).\\
		Aprire il file .exe scaricato e proseguire con l'installazione guidata. Scegliere ``ThymioII package" quando richiesto.
		%TODO verificare
		
	\subsection{MacOS}
	
		Recarsi all'indirizzo \url{https://www.thymio.org/program-2/} e scaricare il software Thymio Suite cliccando sul bottone \texttt{Download} sotto il logo Apple.\\	
		Aprire il file .dmg scaricato e trascinare il file ThymioSuite.app in Applicazioni.
		
	\subsection{Linux}
	
		Recarsi all'indirizzo \url{https://www.thymio.org/help/linux-installation} e seguire le istruzioni per la piattaforma in uso.
			

\section{Collegamento del robot}

	Prima di utilizzare qualsiasi software, è necessario collegare almeno un robot al computer (vedi sezione \ref{multi-robot} per multipli robot). Questo può essere fatto in due modi: tramite cavo o wireless (se supportato).
	
	\subsection{Cavo USB}
	
		Collegare il cavo in dotazione ad una porta libera sulla propria macchina e alla porta microUSB sul retro del Thymio. Il robot si accenderà automaticamente non appena rilevata la connessione.
		
	\subsection{Dongle wireless (se disponibile)}
	
		Collegare il dongle ad una porta libera sulla propria macchina e accendere il robot tenendo premuto il tasto centrale per circa 3 secondi.
	

\section{Software}

	Tutti i programmi necessari per configurare e programmare Thymio sono raccolti in una sola applicazione, Thymio Suite.
	
	\begin{figure}[H]
		\includegraphics[width=\textwidth]{img/thymioSuite.png}
		\caption{La finestra principale di Thymio Suite}
		\label{aseba1}
	\end{figure}
	
	Nota: su MacOS il programma verrà bloccato dal sistema siccome proviene da una fonte esterna (vedi figura \ref{macErr}). Cliccare su Annulla, in seguito recarsi in Impostazioni di Sistema, sotto Sicurezza e Privacy e cliccare su Apri Comunque. Ora è possibile avviare il software, cliccando un'ultima volta su Apri quando richiesto. Altri avvisi appariranno lanciando Blockly e le altre applicazioni, ma è sufficiente cliccare su Apri.
		
	\begin{figure}[H]
		\centering
			\includegraphics[width=0.7\textwidth]{img/macWarn2.png}
			\caption{Il messaggio di avviso di MacOS e le Impostazioni di Sistema}
			\label{macErr}
	\end{figure}
		
	\subsection{Aseba Studio}
	
		Aseba Studio è un software per programmare direttamente i robot Thymio, pensato per utenti più esperti. Per utilizzarlo è sufficiente lanciare Thymio Suite e cliccare su Aseba Studio. Scegliere un robot dalla lista e premere su \texttt{Programma con Aseba Studio}.
		
		\begin{figure}[H]
			\includegraphics[width=\textwidth]{img/asebaStudio.png}
			\caption{La finestra principale di Aseba Studio}
			\label{aseba1}
		\end{figure}		

	\subsection{Blockly}

		Blockly è un software per programmare in maniera intuitiva. Per utilizzarlo è sufficiente lanciare Thymio Suite e cliccare su Blockly. Scegliere un robot dalla lista e premere su \texttt{Programma con Blockly}.
		
		\begin{figure}[H]
			\includegraphics[width=\textwidth]{img/blockly.png}
			\caption{La finestra principale di Blockly}
			\label{aseba1}
		\end{figure}
		
	\subsection{VPL}
	
		VPL (Visual Programming Language) è un software semplificato pensato per i più giovani. Per utilizzarlo è sufficiente lanciare Thymio Suite e cliccare su VPL. Scegliere un robot dalla lista e premere su \texttt{Programma con VPL}.
		
		%TODO immagine
		
	\subsection{Scratch}
	
		%TODO
		
		%TODO immagine
		
	\subsection{Thymio Simulator}
	
		No Thymio? No problem! Esiste un simulatore previsto per emulare uno o più robot sul proprio computer, incluso con il software scaricato al punto \ref{installation}.
		
		%TODO istruzioni
	
		\url{https://www.thymio.org/thymio-simulator/}
		
		
\section{Esempi di codice}

	\subsection{USI Showroom}
		
		Al seguente indirizzo sono disponibili diversi file con esempi di semplici programmi appositamente prodotti dall'USI:
		
		\url{https://github.com/USI-Showroom/thymio/examples}
		
		
		%TODO elenco e spiegazioni
		
		%TODO controllare un robot con un altro
		
	
	\subsection{Risorse ufficiali}
	
		Ai seguenti indirizzi sono disponibili diversi tutorial ed esempi ufficiali di codice:
	
		\url{http://wiki.thymio.org/en:creations}
		
		\url{https://github.com/Mobsya/thymio-programming-exercises}
		
		\url{https://github.com/Mobsya/thymio-vpl-tutorial}
	
	
\section{Comunicazione fra robot}\label{network}

	È possibile far comunicare due robot tramite infrarossi (IR). Siccome la comunicazione avviene tramite i sensori orizzontali (cinque anteriori e due posteriori), i robot devono essere in grado di ``vedersi''.
	
	Prima di tutto è necessario attivare la comunicazione IR su entrambi i robot, dopodiché si possono usare i blocchi o i comandi specialmente previsti (rispettivamente su Blockly e Aseba studio) per trasmettere e ricevere i segnali; purtroppo al momento questa funzione non è disponibile in VPL. 
	
	Di seguito un semplice esempio dove il primo robot trasmette un segnale IR e il secondo reagisce cambiando il colore del LED superiore: 
	
	\begin{figure}[H]
		\includegraphics[width=\textwidth]{img/blocklyIR1.png}
		\label{blocklyIR1}
	\end{figure}
		
	\begin{figure}[H]
		\includegraphics[width=\textwidth]{img/blocklyIR2.png}
		\caption{Un semplice esempio di comunicazione fra due robot}
		\label{blocklyIR2}
	\end{figure}
	
	
\section{Collegare multipli robot}\label{multi-robot}

	È possibile collegare diversi robot allo stesso computer e programmarli in maniera individuale.
	
	\begin{figure}[H]
		\includegraphics[width=\textwidth]{img/multiRobot.png}
		\caption{Aseba Studio con due robot collegati}
		\label{multiRobot}
	\end{figure}
	
	%TODO istruzioni
	
	Istruzioni ufficiali (in inglese): \url{http://wiki.thymio.org/en:thymiosettingwireless}

		
\section{Aggiornamento del firmware}

	Recarsi all'indirizzo \url{https://github.com/Mobsya/aseba-target-thymio2/releases} e scaricare il file \texttt{.hex} per la versione più recente. Collegare il robot con il cavo USB e avviare Thymio Firmware Upgrader; scegliere Custom Firmware, selezionare il file appena scaricato e cliccare su Upgrade. 
	
	%TODO aggiornare
	
	\textbf{NON} scollegare o spegnere il robot durante la procedura!
	
\end{document}